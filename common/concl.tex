%% Согласно ГОСТ Р 7.0.11-2011:
%% 5.3.3 В заключении диссертации излагают итоги выполненного исследования, рекомендации, перспективы дальнейшей разработки темы.
%% 9.2.3 В заключении автореферата диссертации излагают итоги данного исследования, рекомендации и перспективы дальнейшей разработки темы.
\begin{enumerate}
  \item Решена задача о сдавливании круглого идеально жесткопластического тонкого слоя в динамической постановке. Найдены аналитические решения для стадий процесса, соответствующих переходу от квазистаческого к динамическому режиму деформирования и развитого динамического деформирования. Анализ напряженно-деформированного состояния на этих стадиях показал качественное изменение функции давления: в ней возникло слагаемое, квадратично зависящее от продольной координаты, причем с увеличением динамичности процесса значимость данного слагаемого растет. Определена область применимости найденных решений и построен явный критерий, устанавливающий зависимость между временем и стадией процесса прессования.  Согласно последнему, не зависимо от малости величины скорости сближения жестких прессующих поверхностей, наступает конечный момент времени, когда влияние динамических слагаемых становится соизмеримым с градиентом напряжения.
  \item Решена задача о сдавливании цилиндрического идеально жесткопластического тонкого слоя в динамической постановке. В данной задаче естественно возникает дополнительный параметр, отвечающий за форму сжимающих цилиндров. Были рассмотрены случаи, когда радиусы цилиндров имеют тот же порядок, что и толщина слоя, когда радиусы цилиндров порядка длины образующей и когда случай ``промежуточного'' порядка. Для указанных случаев найдены аналитические решения для стадий процесса, соответствующих переходу от квазистаческого к динамическому режиму деформирования и развитого динамического деформирования. Анализ напряженно-деформированного состояния на этих стадиях показал качественное изменение функции давления: в ней возникло слагаемое, квадратично зависящее от продольной координаты, причем с увеличением динамичности процесса значимость данного слагаемого растет. Определена область применимости найденных решений и построен явный критерий, устанавливающий зависимость между временем и стадией процесса прессования.  Согласно последнему, не зависимо от малости величины скорости сближения жестких прессующих поверхностей, наступает конечный момент времени, когда влияние динамических слагаемых становится соизмеримым с градиентом напряжения.
  \item Решена задача о сдавливании сферического идеально жесткопластического тонкого слоя при наличии стока в динамической постановке. Найдены аналитические решения для стадий процесса, соответствующих переходу от квазистаческого к динамическому режиму деформирования и развитого динамического деформирования. Анализ напряженно-деформированного состояния на этих стадиях показал качественное изменение функции давления: в ней возникло слагаемое, величина которого описывается выпуклой вверх функцией с максимумом в центре слоя по простиранию, причем с увеличением динамичности процесса значимость данного слагаемого растет. Определена область применимости найденных решений и построен явный критерий, устанавливающий зависимость между временем и стадией процесса прессования.  Согласно последнему, не зависимо от малости величины скорости сближения жестких прессующих поверхностей, наступает конечный момент времени, когда влияние динамических слагаемых становится соизмеримым с градиентом напряжения.
  \item Решена задача о сдавливании вязкопластического тонкого слоя со степенной функцией упрочнения в динамической постановке. Найдены аналитические решения для стадий процесса, соответствующих переходу от квазистаческого к динамическому режиму деформирования и развитого динамического деформирования. Анализ напряженно-деформированного состояния на этих стадиях показал качественное изменение функции давления: в ней возникло слагаемое, квадратично зависящее от продольной координаты, причем с увеличением динамичности процесса значимость данного слагаемого растет. Определена область применимости найденных решений и построен явный критерий, устанавливающий зависимость между временем и стадией процесса прессования.  Согласно последнему, не зависимо от малости величины скорости сближения жестких прессующих поверхностей, наступает конечный момент времени, когда влияние динамических слагаемых становится соизмеримым с градиентом напряжения.
  \item Для выполнения поставленных задач был развит метод асимптотического интегрирования.
  \item Математическое моделирование показало важность учета инерционных сил в тонких слоях. Установлено качественное изменение эпюры давление и увеличение суммарной силы действующей со стороны материала на прессующие поверхности.
\end{enumerate}
