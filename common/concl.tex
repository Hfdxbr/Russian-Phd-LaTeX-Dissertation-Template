%% Согласно ГОСТ Р 7.0.11-2011:
%% 5.3.3 В заключении диссертации излагают итоги выполненного исследования, рекомендации, перспективы дальнейшей разработки темы.
%% 9.2.3 В заключении автореферата диссертации излагают итоги данного исследования, рекомендации и перспективы дальнейшей разработки темы.
\begin{enumerate}
  \item Получены приближенные аналитические решения в задаче о сдавливании круглого идеально жесткопластического тонкого слоя в динамической постановке. Рассмотрены две стадии процесса, соответствующие переходу от квазистатического к динамическому режиму деформирования и развитому динамическому деформированию.
  \item Получены приближенные аналитические решения в задаче о сдавливании цилиндрического идеально жесткопластического тонкого слоя в динамической постановке. В данной задаче естественно возникает дополнительный параметр, отвечающий за соотношение радиусов и длины образующей сжимающих цилиндров. Рассмотрены случаи, когда радиусы цилиндров имеют тот же порядок, что и толщина слоя, когда радиусы цилиндров порядка длины образующей, и случай ``промежуточного'' порядка. Для указанных случаев исследованы две стадии процесса прессования: переход от квазистатического к динамическому режиму деформирования и развитое динамическое деформирование.
  \item Получены приближенные аналитические решения в задаче о сдавливании сферического идеально жесткопластического тонкого слоя при наличии стока в динамической постановке. Показано что процесс прессования разбивается по времени на качественно различные стадии: этап соответствующий переходу от квазистатического к динамическому режиму деформирования и этап развитого динамического деформирования.
  \item Получены приближенные аналитические решения в задаче о сдавливании вязкопластического тонкого слоя со степенной функцией упрочнения в динамической постановке для режимов прессования, соответствующих стадии перехода от квазистатического к динамическому режиму деформирования и стадии развитого динамического деформирования.
  \item Анализ напряженно-деформированного состояния на исследуемых стадиях показал качественное изменение эпюры давление и увеличение суммарной силы действующей со стороны материала на прессующие поверхности: в функции давления возникло зависящее от продольной координаты слагаемое (выпуклая вверх функция с центром в северном полюсе в случае сферического слоя и квадратичная функция в остальных задачах), причем с приближением к моменту ``схлопывания'' слоя вклад данного слагаемого растет.
  \item Определена область применимости найденных решений и построен явный критерий, устанавливающий зависимость между временем и стадией процесса прессования. Согласно последнему независимо от малости постоянной скорости сближения жестких прессующих поверхностей наступает временной интервал, когда влияние динамических слагаемых становится соизмеримым с градиентами напряжений.
  \item Развит метод асимптотического интегрирования для динамических задач пластического течения при прессовании асимптотически тонких слоев.
\end{enumerate}
