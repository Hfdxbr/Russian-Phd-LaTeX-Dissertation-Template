%%% Микротипографика %%%
%\ifnumequal{\value{draft}}{0}{% Только если у нас режим чистовика
%    \usepackage[final]{microtype}[2016/05/14] % улучшает представление букв и слов в строках, может помочь при наличии отдельно висящих слов
%}{}

% Подчеркивание тензоров
\usepackage{accents}
\newcommand{\utilde}[1]{\underaccent{\tilde}{#1}}
\newcommand{\uindex}[2]{\accentset{\{#2\}}{#1}}
\DeclareMathOperator{\sign}{sign}
\DeclareMathOperator{\const}{const}

% Новая строка с переносом знака
\newcommand{\unl}[2][0]{%
\ifnum 1=#1 \right. \fi
\ifnum 2=#1 \right.\right. \fi
\ifnum 3=#1 \right.\right.\right. \fi
\ifnum 4=#1 \right.\right.\right.\right. \fi
\ifnum 5=#1 \right.\right.\right.\right.\right. \fi
#2 \\ #2
\ifnum 1=#1 \left. \fi
\ifnum 2=#1 \left.\left. \fi
\ifnum 3=#1 \left.\left.\left. \fi
\ifnum 4=#1 \left.\left.\left.\left. \fi
\ifnum 5=#1 \left.\left.\left.\left.\left. \fi
}