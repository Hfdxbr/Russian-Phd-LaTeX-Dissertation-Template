\chapter*{Список сокращений и условных обозначений} % Заголовок
\addcontentsline{toc}{chapter}{Список сокращений и условных обозначений}  % Добавляем его в оглавление
% при наличии уравнений в левой колонке значение параметра leftmargin приходится подбирать вручную
\begin{description}[align=right,leftmargin=3.5cm]
\item[\(\varrho\)] плотность материала
\item[\(\sigma_{ij}\)] тензор напряжений
\item[\(s_{ij}\)] компоненты девиатора тензора напряжений
\item[\(p\)] шаровая часть тензора напряжений
\item[\(\sigma_{s}\)] предел текучести материала
\item[\(r, z, \theta\)] цилиндрические координаты
\item[\(r, \theta, \phi\)] сферические координаты
\item[\(\alpha\)] малый геометрический параметр
\item[Eu] число Эйлера
\end{description}
